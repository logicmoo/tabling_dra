\documentclass[12pt,draft]{report}
\usepackage{fancyvrb}
\usepackage{float}
\usepackage{varioref}
\usepackage[nottoc]{tocbibind} % Doesn't play well with package "index" :-(
\usepackage{makeidx,showidx}
\usepackage{index}  % prevents expansion in index entries, needed for progidx

%\proofmodetrue      %  <<<< (index) suppress for the final copy <<<<<<

\newindex{xpreds}{pdx}{pnd}{Index of procedures}


\makeindex

\setlength{\parindent}{2ex}
\setlength{\parskip}{0.5ex}

\floatplacement{listing}{htbp}

%----------------------------------------------------------------------
%  MACROS

% Facilitate transition between article and book:
\newcommand*{\Chapter}[1]{\chapter{#1}}
\newcommand*{\Section}[1]{\section{#1}}
\newcommand*{\Subsection}[1]{\subsection{#1}}
%\newcommand*{\Chapter}[1]{\section{#1}}
%\newcommand*{\Section}[1]{\subsection{#1}}
%\newcommand*{\Subsection}[1]{\subsubsection{#1}}


% Definition of a new concept:
\newcommand*{\defconcept}[1]{\textbf{\emph{#1}}\index{#1|textsl}}
% Ditto if the word is different from the index entry:
\newcommand*{\defconcepti}[2]{\textbf{\emph{#1}}\index{#2|textsl}}

% Definition in the glossary
\newcommand*{\defgloss}[1]{#1\index{#1|textsl}}
% Ditto if the word is different from the index entry:
\newcommand*{\defglossi}[2]{#1\index{#2|textsl}}

% A short form of a tiny \marginpar:
\newcommand*{\mpar}[1]{\marginpar{\tiny#1}}

% A word in the text that should be indexed:
\newcommand*{\Index}[1]{#1\index{#1}}

% A word that is the object of discussion:
\newcommand*{\about}[1]{\emph{#1}}

% A small piece of a concrete program etc.:
\newcommand*{\prog}[1]{\texttt{#1}}

% A ``pattern'', e.g., a generic variable within a concrete call:
\newcommand*{\patt}[1]{\textsl{#1}}

% A reference to an entity in a program:
\newcommand*{\progfrag}[1]{\about{#1}}

% An example of a term in the text:
\newcommand*{\term}[1]{\progfrag{#1}}

% A predicate specification:
\newcommand*{\pred}[1]{\about{#1}}

% An index entry for a procedure:
\newcommand*{\progidx}[1]{\index[xpreds]{#1@\prog{#1}}}

% A reference to a figure:
\newcommand*{\figref}[1]{Fig.~\vref{#1}}

% A reference to a listing:
\newcommand*{\lstref}[1]{Listing~\vref{#1}}

% A reference to a chapter:
\newcommand*{\chapref}[1]{chapter~\ref{#1}}

% A reference to a section:
\newcommand*{\secref}[1]{Sec.~\ref{#1}}

% An indentation
\newcommand*{\ind}{\hbox{\hspace{2em}}}

% Mathcal in normal text:
\newcommand*{\normcal}[1]{\(\mathcal{#1}\)}

% An explanation in Dijsktra-style proof via eqnarray*:
\newcommand*{\expl}[1]{\mbox{[~\emph{#1}~]}}

% Abbreviations for use in tabular:
\newcommand*{\h}{\hspace{1pt}}
\newcommand*{\mc}[3]{\multicolumn{#1}{#2}{#3}}

% The set of natural numbers: must be in math mode!
\newcommand*{\N}{\mathcal{N}}

% The powerset of something: must be in math mode!
\newcommand*{\Power}{\mathcal{P}}

% A binary relation between two arguments: must be in math mode!
% \binrel{a}{R}{c} to get a R b.
\newcommand*{\binrel}[3]{#1\,#2\,#3}

% The implication symbols: must be in math mode!
\newcommand*{\Implies}{\Rightarrow}
\newcommand*{\Implied}{\Lefttarrow}
\newcommand*{\IFF}{\Leftrightarrow}

% The disjunction symbol: must be in math mode!
\newcommand*{\Or}{\vee}

% The conjunction symbol: must be in math mode!
\newcommand*{\And}{\wedge}

% The dot that divides the quantifier and variables and the quantified
% formula: must be in math mode!
\newcommand*{\qdot}{ \;.\;}

% A universally quantified formula: must be in math mode!
\newcommand*{\All}[2]{(\forall #1 \qdot #2)}

% An equivalence class with for this relation, with this representative: must
% be in math mode!
\newcommand*{\EqClass}[2]{[#2]_{#1}}



%----------------------------------------------------------------------
%  ENVIRONMENTS

\floatstyle{boxed}
\newfloat{listing}{htbp}{lst}[chapter]
\floatname{listing}{Listing}

\newenvironment{warning}%
{\begin{quote}\textbf{Warning:}\itshape}%
{\end{quote}}


%----------------------------------------------------------------------
\title{The DRA Interpreter\\
User Manual}

\author{Feliks Klu{\'z}niak\\
  \emph{Applied Logic, Programming Languages and Systems Lab}\\
  \emph{Department of Computer Science}\\
  \emph{University of Texas at Dallas}
}
\date{\today}

\bibliographystyle{plain}


%----------------------------------------------------------------------
\begin{document}

\maketitle

\tableofcontents

\Chapter{Introduction\label{sec:intro}}

This document is a user manual for \about{dra}, an interpreter for tabled
logic programming with coinduction.

The interpreter implements ``top-down tabled programming'' via so called
``Dynamic Reordering of Alternatives'' \cite{guo-gupta-dra}.  It also
supports ``co-logic programming'', i.e., logic programs that contain
coinductive predicates \cite{coinductive}, \cite{co-LP}.

Appart from support for coinduction, there are two significant changes with
respect to the original description \cite{guo-gupta-dra}:
\begin{enumerate}

\item A tabled goal will never produce the same answer twice.

  More specifically: two answers will never be variants of each
  other.\footnote{
  Please note that in this document \about{goal} means an instance of a
  procedure call.}

\item By default, new answers for a tabled goal will be produced before old
  answers.  The user can reverse the order by means of a directive
  (\prog{:-~old\_first.}).

  A ``new answer for a tabled goal'' is an answer that has not yet been seen
  (and tabled) for a variant of the goal.

  The default behaviour is intended to help computations converge more
  quickly.  The user is given an option to change it, because some predicates
  may produce a very large (even infinite) set of answers on backtracking,
  and the application might not require those answers.
\end{enumerate}

\Chapter{Starting the interpreter\label{chap:starting}}

The interpreter is written in Prolog.  It is distributed in source
form.\footnote{%
  Please see the ``README'' files in the distribution tree: they will help you
  find your way around.}

The interpreter is known to run on Eclipse 6.0 and Sicstus 4.0.  If you plan
to run programs that take advantage of coinductive programming, you will
probably do better with Sicstus, which has much better support for cyclic
terms.

For both these systems, the simplest way to proceed is:
\begin{enumerate}
\item start your logic programming system;
\item type in the following directive:\\
  \ind\prog{:-~[~'}\about{Path}\prog{/tabling/dra'~]}\\
  where \about{Path} is the path to the root of the distribution tree.
\end{enumerate}

It may well be that things have been installed differently on your site.
This might be because the interpreter has been modified to run with a
different Prolog system, or because an immediately-loadable version has been
made available in some standard directory. The person responsible for the
local installation of the interpreter will provide you with more details.


\bibliography{bibliography}
\printindex
\end{document}
