\documentclass[12pt,final]{report}
\usepackage{fancyvrb}
\usepackage{float}
\usepackage{varioref}
\usepackage[nottoc]{tocbibind} % Doesn't play well with package "index" :-(
\usepackage{makeidx}
%\usepackage{index}  % prevents expansion in index entries, needed for progidx
                     % Makes hyperref for index go awry, so decided to use
                     % explicit protect instead: turned out not to be neeeded

%\proofmodetrue      %  <<<< (index) suppress for the final copy <<<<<<


\usepackage{parskip}[2001/04/09]
%\setlength{\parindent}{0pt}
%\setlength{\parskip}{2ex}

\usepackage[colorlinks]{hyperref}

\makeindex

%----------------------------------------------------------------------
%  MACROS

% Facilitate transition between article and book:
\newcommand*{\Chapter}[1]{\chapter{#1}}
\newcommand*{\Section}[1]{\section{#1}}
\newcommand*{\Subsection}[1]{\subsection{#1}}
%\newcommand*{\Chapter}[1]{\section{#1}}
%\newcommand*{\Section}[1]{\subsection{#1}}
%\newcommand*{\Subsection}[1]{\subsubsection{#1}}

% Three asterisks break the text up where even a subsection is not merited:
% use this as a separate paragraph:
\newcommand*{\Breakup}{\[\ast\ast\ast\]}

% Definition of a new concept:
\newcommand*{\Defconcept}[1]{\emph{#1}\index{#1}}
% Ditto if the word is different from the index entry:
\newcommand*{\Defconcepti}[2]{\emph{#1}\index{#2}}

% A short form of a tiny \marginpar:
\newcommand*{\mpar}[1]{\marginpar{\tiny#1}}

% A word in the text that should be indexed:
\newcommand*{\Index}[1]{#1\index{#1}}

% A word that is the object of discussion:
\newcommand*{\about}[1]{\emph{#1}}

% A small piece of a concrete program etc.:
\newcommand*{\prog}[1]{\texttt{#1}}

% A ``pattern'', e.g., a generic variable within a concrete call:
\newcommand*{\patt}[1]{\textit{#1}}

% A reference to an entity in a program:
\newcommand*{\progfrag}[1]{\about{#1}}

% An example of a term in the text:
\newcommand*{\term}[1]{\progfrag{#1}}

% A predicate specification:
\newcommand*{\pred}[1]{\about{#1}}

% Ditto within an index entry:
\newcommand*{\predidx}[1]{\about{#1}\index{#1@\pred{#1}}}

% Ditto without putting into the text:
\newcommand*{\predidxonly}[1]{\index{#1@\pred{#1}}}

% An index entry for a procedure:
\newcommand*{\progidx}[1]{\prog{#1}\index{#1@\prog{#1}}}

% Ditto without putting into the text:
\newcommand*{\progidxonly}[1]{\index{#1@\prog{#1}}}

% A reference to a figure:
\newcommand*{\Figref}[1]{Fig.~\vref{#1}}

% A reference to a chapter:
\newcommand*{\Chapref}[1]{chapter~\ref{#1}}

% A reference to a section:
\newcommand*{\Secref}[1]{Sec.~\protect\ref{#1}}

% A reference to a page:
\newcommand*{\Pageref}[1]{p.~\pageref{#1}}

% An indentation
\newcommand*{\ind}{\hbox{\hspace{2em}}}


%----------------------------------------------------------------------
%  ENVIRONMENTS

% A warning:
\newenvironment{Warning}%
{\begin{quote}\textbf{Warning:}\itshape}%
{\end{quote}}

% Itemize with no topsep:
\newenvironment{Itemize}%
{\begin{list}{$\bullet$}%
    { \setlength{\topsep}{0pt}%
}}%
{\end{list}}

% A lightweight itemize:
\newenvironment{LightItemize}%
{\begin{list}{--}%
    { \setlength{\itemsep}{0.1ex}%
      \setlength{\topsep}{0pt}%
}}%
{\end{list}}

% Enumerate with no topsep:
\newcounter{Enumcnt}
\newenvironment{Enumerate}
               {\begin{list}{\arabic{Enumcnt}.}%
                   { \setlength{\topsep}{0pt}%
                     \usecounter{Enumcnt}%
               }}%
               {\end{list}}

% A lightweight enumerate, labeled with (a), {b):
\newcounter{lightenumcnt}
\newenvironment{LightEnumerate}
               {\begin{list}{(\alph{lightenumcnt})}%
                   {\setlength{\itemsep}{0pt}%
                     \setlength{\topsep}{0pt}%
                     \settowidth{\labelwidth}{(m)}%
                     \usecounter{lightenumcnt}%
               }}%
               {\end{list}}

%----------------------------------------------------------------------
\title{The DRA Interpreter\\
User Manual}

\author{Feliks Klu{\'z}niak\\
  \emph{Applied Logic, Programming Languages and Systems Lab}\\
  \emph{Department of Computer Science}\\
  \emph{University of Texas at Dallas}
}
\date{\small\today}

\bibliographystyle{plain}


%----------------------------------------------------------------------
\begin{document}

\maketitle

%%% reverse of title page
\thispagestyle{empty}
\setcounter{page}{0}

\fbox{\small
  \begin{minipage}{\textwidth}
    NOTICE:\\

    \copyright 2009 University of Texas at Dallas

    \mbox{}

    Developed at the Applied Logic, Programming Languages and Systems (ALPS)
    Laboratory at UTD by Feliks Klu{\'z}niak.

    \mbox{}

    Permission is granted to modify this text, and to distribute its original
    or modified contents for non-commercial purposes, on the condition that
    this notice is included in all copies in its original form.

    \mbox{}

    THE SOFTWARE IS PROVIDED ``AS IS'', WITHOUT WARRANTY OF ANY KIND, EXPRESS
    OR IMPLIED, INCLUDING BUT NOT LIMITED TO THE WARRANTIES OF
    MERCHANTABILITY, FITNESS FOR A PARTICULAR PURPOSE, TITLE AND
    NON-INFRINGEMENT. IN NO EVENT SHALL THE COPYRIGHT HOLDERS OR ANYONE
    DISTRIBUTING THE SOFTWARE BE LIABLE FOR ANY DAMAGES OR OTHER LIABILITY,
    WHETHER IN CONTRACT, TORT OR OTHERWISE, ARISING FROM, OUT OF OR IN
    CONNECTION WITH THE SOFTWARE OR THE USE OR OTHER DEALINGS IN THE
    SOFTWARE.
\end{minipage}
}

\vfill  %%%%%%%%%

{\footnotesize
  All comments, queries and suggestions about this manual or the software
  are welcome. The author's e-mail address is
  \prog{feliks.kluzniak@utdallas.edu}.}
%%% end reverse of title page

\tableofcontents

\Chapter{Introduction\label{sec:intro}}

This document is a user manual for \about{dra}, an interpreter for tabled
logic programming with coinduction.

The interpreter implements ``top-down tabled programming'' via so called
``Dynamic Reordering of Alternatives'' \cite{guo-gupta-dra}.  It also
supports ``co-logic programming'', i.e., logic programs that contain
coinductive predicates \cite{coinductive}, \cite{co-LP}.

Appart from support for coinduction, there are two significant changes with
respect to the original description \cite{guo-gupta-dra}:
\begin{enumerate}

\item A tabled goal will never produce the same answer twice.

  More specifically: two answers will never be variants of each
  other.\footnote{
  Please note that in this document \about{goal} means an instance of a
  procedure call.}

\item By default, new answers for a tabled goal will be produced before old
  answers.  The user can reverse the order by means of a directive
  (\prog{:-~old\_first.}).

  A ``new answer for a tabled goal'' is an answer that has not yet been seen
  (and tabled) for a variant of the goal.

  The default behaviour is intended to help computations converge more
  quickly.  The user is given an option to change it, because some predicates
  may produce a very large (even infinite) set of answers on backtracking,
  and the application might not require those answers.
\end{enumerate}

\Chapter{The interpreted programs\label{chap:programs}}



%-------------------------------------------------------------------------------
\Section{Limitations\label{sec:limitations}}

The interpreter does not support full Prolog.  Here are the main limitations
of the interpreted language:
\begin{Enumerate}

\item
  The interpreted program must not contain cuts\index{cut} (i.e., occurrences
  of \pred{!/2}\index{"!/2@\pred{"!/2}}).  Use of the conditional
  construct\index{conditional construct} is permitted, as is the use of
  \predidx{once/1}.

\item
  The interpreted program must not contain variable literals\index{variable
    literal}.  It may contain invocations of \predidx{call/1}, but if the
  argument of \pred{call/1} is not properly instantiated at runtime, you will
  get an error message and the interpreter will quit.\footnote{
    In some cases the interpreter can verify beforehand (i.e., at
    ``compile-time'') that the argument of an occurrence of \pred{call/1}
    cannot be instantiated at run-time, and it will then raise a fatal error.
    The check is quite conservative, so the absence of such an error message
    does not mean that the program is safe in that respect.}

\item
  The repertoire of built-in predicates\index{built-in predicates} recognized
  by the interpreter is somewhat limited.  This is done by design, mostly to
  facilitate porting to different Prolog systems.

  The recognized built-ins are declared in the file \prog{dra\_builtins.pl},
  and new declarations can be added as the need arises.  For most built-ins
  just adding another line to the file will suffice, but a few might
  require special treatment by the interpreter.\footnote{
    Having a file wherein you specify the names of built-in predicates you
    actually want to use does have its advantages.  Some logic programming
    systems (e.g., \Eclipse{}) support a very extensive set of libraries that
    define built-in predicates whose names are treated as reserved even if
    you don't use the libraries.  As a result, many names that you might
    reasonably want to use in your programs are not available to you.}
\end{Enumerate}

If these limitations seem too strict, you may in some cases get around them
by separating your program into two layers: see \Secref{sec:support}.


%-------------------------------------------------------------------------------
\Section{The notion of ``support''\label{sec:support}}

The interpreter provides you with an opportunity to divide your program into
two layers: an upper layer which makes use of the special facilities provided
by the interpreter (i.e., tabling and/or coinduction), and a lower layer of
``support'' software that requires only standard Prolog.  This can be useful
for increasing efficiency: the support layer will be compiled just as all
other ``normal'' Prolog programs.  An additional advantage is that the
support layer can use the full range of built-in predicates available in the
host logic programming system, and in particular the cut.

The interface between the two layers consists of a handful of entry-point
predicates, each of which is  declared by a directive similar to the
following one:\\
\ind\prog{:- support check\_consistency/1.}%
\label{dir:support}\progidxonly{support}\\
Please note that this directive cannot be entered interactively: it must be
included in the text of the upper layer part of your program.

The support declaration means that the metainterpreter should treat the
declared predicate as a built-in, i.e., just let Prolog execute it.

The support layer cannot invoke the upper layer, so there is no need to
declare those predicates in the support layer that are not directly invoked
by the upper layer.

Predicates that are declared as ``support'' (and those that are---directly or
indirectly---called by them) must be defined in other files.
To compile and load such a file, use the following directive in the text of
your program:\\
\ind\prog{:- load\_support(~\patt{filename}~).}%
\label{dir:load-support}\progidxonly{load\_support}\\
In this context, the default extension of the \patt{filename} will be the
default extension used by the host logic programming system for names
of files that contain Prolog code.%
\index{default extension}%
\index{extension of file name!default}%
\index{file!name!default extension}



%-------------------------------------------------------------------------------
\Section{Declaring ``entry points''\label{sec:entry-points}}

Before execution begins, the interpreted program is subjected to a number of
sanity checks.  One of these is a check whether every defined predicate is
actually called from somewhere (i.e., whether there is no dead code).

Since it is not unusual for a program to contain a handful of such predicates
on purpose (they are intended as ``entry points'' that are to be invoked from
a query),
the user can declare them by using a directive similar to the following:\\
\ind\prog{:-~top~p/1,~q/2.}\label{dir:top}\progidxonly{top}\\
The declaration is given only to suppress warnings.  However, it is an error
for an undefined predicate or a support predicate to be so declared.



%-------------------------------------------------------------------------------
\Section{Declaring dynamic predicates\label{sec:dynamic}}

To declare a predicate whose clauses are asserted and/or retracted by the
interpreted program, use\index{predicate!dynamic}\\
\ind\prog{:-~ dynamic~p/k.}\label{dir:dynamic}\progidxonly{dynamic}



%-------------------------------------------------------------------------------
\Section{Hooks\label{sec:hooks}}\index{hook}

The program may contain clauses that modify the definition of the
interpreter's predicate \predidx{essence\_hook/2} (the clauses will be
asserted at the front of the predicate, and will thus override the default
definition for some cases).  The interpreter's default definition is\\
\ind\prog{essence\_hook(~T,~T~).}

This predicate is invoked, in certain contexts, when:
\begin{LightItemize}
  \item
    two terms are about to be compared (either for equality or to check
    whether they are variants of each other);
  \item
    an answer is tabled;
  \item
    an answer is retrieved from the table.
\end{LightItemize}

The primary intended use is to allow suppression of arguments that carry only
administrative information and that may differ in two terms that are
considered to be ``semantically'' equal or variants of each other.

For example, the presence of\\
\ind\prog{essence\_hook(~p(~A,~B,~\_~),~~p(~A,~B~)~).}\\
will result in \prog{p(~a,~b,~c~)} and \prog{p(~a,~b,~d~)} being treated as
identical: each of them will be translated to \prog{p(~a,~b~)} before
comparison.

\begin{Warning}
This facility should be used with the utmost caution, as it may drastically
affect the semantics of the interpreted program in a fashion that could be
hard to understand for someone who is not familiar with the details of the
interpreter.
\end{Warning}

\Chapter{Running a program\label{chap:running}}

Once you have loaded the interpreter into your logic programing system, you
may want to run a program in the interpreter. This is done by
writing\\
\ind\prog{prog(~}\patt{filename}\prog{~ ).}\progidx{prog}%
\footnote{
  If you are running in Eclipse, and have not imported the module \about{dra}
  (as explained on \vpageref{import-dra}), you must write \prog{dra:prog}
  instead of \prog{prog}.
}\\
\patt{filename} should be the name of the file that contains your program.
If the name has no extension, the default extension will be \prog{.tlp}.  If
the name should have a different extension, you must type in the entire name,
ecnclosed in single quotes, e.g.,\\
\ind\prog{prog(~'myfile.pl'~ ).}\\
Quotes must also be used if the file is not in the current directory and you
are providing an absolute or relative path.

As the file is being read and loaded, directives and queries are interpreted
on-the-fly. Each query is evaluated to give all solutions (i.e., as if the
user kept responding with a semicolon): to avoid that use the built-in
predicate \pred{once/1}.

After the file is loaded (and all the directives and queries it contains are
executed), interactive mode is started.  This is very much like the usual
top-level loop, except that it is the associated interpreter -- and not the
underlying logic programming system -- that is used to evaluate queries and
directives.

Please note that in the interactive mode one cannot input more than one term
per line.


To just enter interactive mode invoke\\
\ind\prog{top}.\progidx{top}%
\footnote{Again, \prog{dra:top} in Eclipse, if you have not imported
  \about{dra}.}\\
 The interpreter does not allow you to input clauses directly from your
 terminal, but this facility may be useful if you have exited interactive
 mode (see below) or interrupted the execution of the interpreter: the
 program that was most recently loaded is still there.

To exit interactive mode enter the end of file character
(\about{Ctrl-D}),%
\footnote{
  \about{Ctrl-D} appears not to work with tkeclipse.}
or just write\\
\ind\prog{quit.}

You should be aware that loading a program obliterates all traces of
previously loaded programs, including the contents of the answer table.  So
if you are interested in re-running your program from scratch (so that it
does not take advantage of answers that were already tabled), you should load
it again.


When a query succeeds, the bindings of variables should be printed upto a
certain maximum depth.  The default value in the distributed version of the
interpreter is 10.  The maximum depth can be changed from the interpreted
program (or interactively from the top-level) by invoking\\
\ind\prog{set\_print\_depth(~}\patt{N}\prog{~ )}\progidx{set\_print\_depth}

where \patt{N} is a positive integer.

Please note that with some Prolog implementations this might not prevent a
loop if the printed term is cyclic (as will often happen for coinductive
programs).

Note also that the foregoing does not apply to invocations of built-in
predicates in the interpreted program.  It is up to the user to apply the
built-in appropriate for the host logic programming system.  For example, in
the case of Sicstus, use \prog{write\_term(~T,~[~max\_depth(~10~)~]~)} rather
than just \prog{write( T )}, especially if you expect the instantiation of
\prog{T} to be cyclic.

\chapter*{Summary of directives\label{directives}}%
\addcontentsline{toc}{chapter}{Summary of directives}

Many of the directives take an argument specified as \patt{PredSpec}.  This can
take three forms:
\begin{LightEnumerate}
\item
  A predicate specification written as \pred{name/arity}: for example
  \prog{foo/3} (in the short descriptions below we will assume this is the form
  that is used);
\item
  A sequence of such specifications, separated by commas: for example
  \prog{p/2,~q/1,~r/3};
\item
  The word \prog{all}, which specifies all predicates.
\end{LightEnumerate}
If the same kind of directive occurs a number of times, specifying different
predicates, the results are cumulative.  In particular, \prog{all} subsumes all
other predicate specifications.

\newlength{\DescWidth}
\setlength{\DescWidth}{16em}
\begin{tabular}{llr}
\emph{Directive}      & \emph{Short description}   \\

\prog{:-~[~\patt{filename}~].}
                   & \parbox[t]{\DescWidth}{
                        load a part of the program (\Pageref{dir:include})}\\

\prog{:-~answers(~\patt{Goal},~\patt{Pattern}~).}
                  & \parbox[t]{\DescWidth}{
                       inspect the answer table (\Pageref{dir:answers})}\\

\prog{:-~coinductive~\patt{PredSpec}.}
                   & \parbox[t]{\DescWidth}{
                       predicate is coinductive (\Pageref{dir:tabled})}\\

\prog{:-~dynamic~\patt{PredSpec}.}
                  & \parbox[t]{\DescWidth}{
                        predicate is dynamic (\Pageref{dir:dynamic})}\\

\prog{:- load\_support(~\patt{filename}~).}
                   & \parbox[t]{\DescWidth}{
                      load (a part of) the support layer
                                                (\Pageref{dir:load-support})}\\

\prog{:-~old\_first.} &  \parbox[t]{\DescWidth}{
                                 change the order in which results are produced
                                 (\Pageref{dir:old-first})
                                 }\\

\prog{:- support~\patt{PredSpec}}
                   & \parbox[t]{\DescWidth}{
                       predicate is an entry point to the support layer
                                                   (\Pageref{dir:support})}\\

\prog{:-~tabled~\patt{PredSpec}.}
                   & \parbox[t]{\DescWidth}{
                            predicate is tabled (\Pageref{dir:tabled})}\\

\prog{:-~top~\patt{PredSpec}}
                   & \parbox[t]{\DescWidth}{
                        predicate is an entry point (\Pageref{dir:top})}\\

\prog{:-~trace~\patt{PredSpec}.}
                  & \parbox[t]{\DescWidth}{
                        trace the predicate (\Pageref{dir:trace})}
\end{tabular}


\newpage
\bibliography{bibliography}

\newpage
\printindex
\end{document}
